\documentclass[uplatex,a4j,12pt]{jsarticle}

\renewcommand{\baselinestretch}{0.8}
% 縦横のサイズ調整
\usepackage[margin=15mm]{geometry}
\usepackage[dvipdfmx]{graphicx}
\usepackage{latexsym}
\usepackage{bmpsize}
\usepackage{url}
\usepackage{comment}
\begin{document}


\title{\vspace{-10mm}InputMethodKitとTCAを使ったmacOS上で動作するIMEの開発}
\author{Tatsumi0000}
\date{}
\maketitle


\section{はじめに}
課題とか、背景とか、目的とか、なんか書く。
第2章では~~~を第3章~~~について説明します。\cite{Hoge}の引用です。
% \begin{figure}[t]
%     \begin{center}
%         \includegraphics[width=7cm]{image/syokuji_computer.png}
%         \caption{パソコンの前でご飯を食べる人のイラスト}
%         \label{fig:syokuji_computer}
%     \end{center}
% \end{figure}

% パソコンの前でご飯を食べることはよくある。パソコンの前でご飯を食べる人のイラストを図\ref{fig:syokuji_computer}に示す。
% このイラストは、規約の範囲内であれば、個人、法人、商用、非商用問わず無料で利用できることでおなじみの、{\bf かわいいフリー素材 いらすとや}\cite{Hoge}より引用した。

\section{今回開発したRaelizeの構成}
今回開発したRaelizeの構成について述べます。

\section{InputMethodKitとは}
InputMethodKitについて述べます。

\subsection{IMKInputController}
IMKInputControllerについて述べます。

\section{The Composable Architecture}
The Composable Architecture(TCA)について述べます。

\section{InputMethodKitとTCAを使ったIME開発}
InputMethodKitとTCAを使ったIME開発について述べます。

\section{マルチモジュール構成とfastlaneを用いた効率的なデバッグ}
マルチモジュール構成とfastlaneを用いた効率的なデバッグについて述べます。

\section{おわりに}

%参考文献
\bibliographystyle{plain} 
\bibliography{iosdc2024,bib} 

\end{document}
